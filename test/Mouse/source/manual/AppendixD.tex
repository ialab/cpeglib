\newpage
%HHHHHHHHHHHHHHHHHHHHHHHHHHHHHHHHHHHHHHHHHHHHHHHHHHHHHHHHHHHHHHHHHHHHHHHHHH

\section{Appendix: \Mouse\ tools\label{tools}}

%HHHHHHHHHHHHHHHHHHHHHHHHHHHHHHHHHHHHHHHHHHHHHHHHHHHHHHHHHHHHHHHHHHHHHHHHHH

\fbox{ \textbf{mouse.Generate}\textit{ arguments}\prop}

\medskip
Generates parser.\newline
The \textit{arguments} are specified as options according to POSIX syntax:
%
\ul
\item[\tx{-G} \textit{filename}]\ \newline
    Identifies the file containing your grammar. Mandatory.\newline
    The \textit{filename} need not be a complete path, 
    just enough to identify the file\newline 
    in current environment.
    It should include file extension, if any. 

\item[\tx{-D} \textit{directory}]\ \newline
    Identifies target directory to receive the generated file(s).\newline
    Optional. If omitted, files are generated in current work directory.\newline 
    The \textit{directory} need not be a complete path, just enough to identify
    the directory\newline 
    in current environment. The directory must exist.

\item[\tx{-P} \textit{parser}]\ \newline
    Specifies name of the parser to be generated. Mandatory.\newline
    Must be an unqualified class name.
    The tool generates a file named "\textit{parser}\tx{.java}"
    in the target directory,
    The file contains definition of Java class \textit{parser}.
    If target directory already contains a file "\textit{parser}\tx{.java}",
    the file is replaced without a warning,

\item[\tx{-S} \textit{semantics}]\ \newline
    Indicates that semantic actions are methods in the Java class \textit{semantics}.\newline
    Mandatory if your grammar specifies semantic actions.
    Must be an unqualified class name.

\item[\tx{-p} \textit{package}]\ \newline
    Generate parser as member of package \textit{package}.\newline  
    The semantics class, if specified, is assumed to belong to the same package.\newline
    Optional. If not specified, both classes belong to unnamed package.\newline
    The specified package need not correspond to the target directory.

\item[\tx{-r} \textit{runtime-package}]\ \newline
    Generate parser that will use runtime support package \textit{runtime-package}.\newline  
    Optional. If not specified, \tx{mouse.runtime} is assumed.

\item[\tx{-s}]
    Generate skeleton of semantics class. Optional.\newline
    The skeleton is generated as file "\textit{semantics}\tx{.java}" in the target directory,
    where \textit{semantics} is the name specified by \tx{-S} option.
    The option is ignored if \tx{-S} is not specified.
    If target directory already contains a file "\textit{semantics}\tx{.java}",
    the tool is not executed, to prevent accidental destruction of your semantics class.    

\item[\tx{-M}] Generate memoizing version of the parser.

\item[\tx{-T}] Generate instrumented test version of the parser.

\eul

(Options \tx{-M} and \tx{-T} are mutually exclusive.)

%----------------------------------------------------------------------
\newpage
\fbox{ \textbf{mouse.TestPEG}\textit{ arguments}\prop}

\medskip
Checks your grammar without generating parser.\newline
The \textit{arguments} are specified as options according to POSIX syntax:
%
\ul
\item[\tx{-G} \textit{filename}]\ \newline
    Identifies the file that contains your grammar. Mandatory.\newline
    The \textit{filename} should include any extension.\newline 
    Need not be a complete path, just enough to identify the file
    in your environment.

\item[\tx{-D}]Display the grammar. Optional.\newline
    Shows the rules and subexpressions together with their attributes according to Ford.\newline
    (\tx{0}=may consume empty string, \tx{1}=may consume nonempty string, 
    \tx{f}= may fail, \tx{!WF}=not well-formed.)

\item[\tx{-C}] 
    Display the grammar in compact form: without duplicate subexpressions. Optional.

\item[\tx{-R}] 
    Display only the rules. Optional.
\eul

(Obviously, \tx{-C}, \tx{-D}, and \tx{-R} are mutually exclusive.)

%----------------------------------------------------------------------
\bigskip
\fbox{ \textbf{mouse.TryParser}\textit{ arguments}\prop}

\medskip
Runs the generated parser.\newline
The \textit{arguments} are specified as options according to POSIX syntax:
%
\ul

\item[\tx{-P} \textit{parser}]\ \newline
    Identifies the parser. Mandatory.\newline
    \textit{parser} is the class name, fully qualified with package name, if applicable.
    The class must reside in a directory corresponding to the package.

\item[\tx{-f} \textit{filename}]\ \newline
    Apply the parser to file \textit{filename}. Optional.\newline
    The \textit{filename} should include any extension.\newline 
    Need not be a complete path, just enough to identify the file
    in your environment.

\item[\tx{-F} \textit{list}]\ \newline
    Apply the parser separately to each file in a list of files. Optional.\newline
    The \textit{list} identifies a text file containing one fully qualified file name per line.
    The \textit{list} itself need not be a complete path, just enough to identify the file
    in your environment.

\item[\tx{-m} \textit{n}]\ \newline
    Amount of memoization. Optional. Applicable only to a parser generated with option \tx{-M}.\newline 
    \textit{n} is a digit from 1 through 9 specifying the number of results to be cached. 
    Default is no memoization.
    
\item[\tx{-T} \textit{string}]\ \newline
    Tracing switches. Optional.\newline
    The \textit{string} is assigned to the \tx{trace} field in your semantics object,
    where you can use it to activate any trace you have programmed there.

\item[\tx{-t}] Display timing information.\newline
    Note that the precision of time measurements corresponds to the granularity of the system's clock.
    Also, that Java's execution times may vary.

\eul

Options \tx{-f} and \tx{-F} are mutually exclusive.

If you do not specify \tx{-f} or \tx{-F}, 
the parser is executed interactively, prompting for input by printing "\tx{>}".
It is invoked separately for each input line after you press "Enter".
You terminate the session by pressing "Enter" directly at the prompt.

%----------------------------------------------------------------------
\newpage
\fbox{ \textbf{mouse.TestParser}\textit{ arguments}\prop}

\medskip
Runs the instrumented parser (generated with option \tx{-T}),
and prints information about its operation,
as explained in Sections \ref{back}, 
\ref{packrat}, and~\ref{back2}.
The \textit{arguments} are specified as options according to POSIX syntax:
%
\ul

\item[\tx{-P} \textit{parser}]\ \newline
    Identifies the parser. Mandatory.\newline
    \textit{parser} is the class name, fully qualified with package name, if applicable.
    The class must reside in a directory corresponding to the package.

\item[\tx{-f} \textit{filename}]\ \newline
    Apply the parser to file \textit{filename}. Optional.\newline
    The \textit{filename} should include any extension.\newline 
    Need not be a complete path, just enough to identify the file
    in your environment.

\item[\tx{-F} \textit{list}]\ \newline
    Apply the parser separately to each file in a list of files. Optional.\newline
    The \textit{list} identifies a text file containing one fully qualified file name per line.
    The \textit{list} itself need not be a complete path, just enough to identify the file
    in your environment.

\item[\tx{-m} \textit{n}]\ \newline
    Amount of memoization. Optional.\newline 
    \textit{n} is a digit from 1 through 9 specifying the number of results to be retained.
    Default is no memoization.

\item[\tx{-T} \textit{string}]\ \newline
    Tracing switches. Optional,\newline
    The \textit{string} is assigned to the \tx{trace} field in your semantics object,
    where you can use it to trigger any trace you have programmed there.
    In addition, presence of certain letters in \textit{string}
    activates traces in the parser:\newline
    \tx{r} -- trace execution of parsing procedures for rules.\newline
    \tx{i} -- trace execution of parsing procedures for inner expressions.\newline
    \tx{e} -- trace error information.
        
\item[\tx{-d}]
    Show detailed statistics for procedures involved in backtracking - rescan - reuse. Optional.
        
\item[\tx{-D}]
    Show detailed statistics for all invoked procedures. Optional.
    
\item[\tx{-C} \textit{filename}]\ \newline
    Write all statistics as comma-separated values (CSV) to file \textit{filename},
    rather than to System.out.\newline
    Optional; can only be specified with \tx{-F}.\newline
    The \textit{filename} should include any extension.\newline
    Need not be a complete path, just enough to identify the file
    in your environment.\newline
    If the file \textit{filename} exists, it is overwriten.
    Otherwise the file is created.

\item[\tx{-t}] Display timing information.

\eul

Options \tx{-f} and \tx{-F} are mutually exclusive, as well as \tx{-d} and \tx{-D}.

If you do not specify \tx{-f} or \tx{-F}, 
the parser is executed interactively, prompting for input by printing "\tx{>}".
It is invoked separately for each input line after you press "Enter".
You terminate the session by pressing "Enter" directly at the prompt.


%----------------------------------------------------------------------
\bigskip
\fbox{ \textbf{mouse.MakeRuntime}\textit{ arguments}\prop}

\medskip
Re-package \Mouse\ runtime support:
write out Java source files for runtime-support classes\newline
using specifed package name instead of \tx{mouse.runtime}.\newline
The \textit{arguments} are specified as options according to POSIX syntax:
%
\ul
\item[\tx{-r} \textit{runtime-package}]\ \newline
    Generate files with package name \textit{runtime-package}. Mandatory.  
    
\item[\tx{-D} \textit{directory}]\ \newline
    Identifies target directory to receive the files.\newline
    Optional. If omitted, files are written to the current work directory.\newline 
    The \textit{directory} need not be a complete path, just enough to identify
    the directory\newline 
    in current environment. The directory must exist.\newline
    The directory name need not correspond to the package name.

\eul


%----------------------------------------------------------------------
\bigskip
\fbox{ \textbf{mouse.ExplorePEG} \tx{-G} \textit{filename}\prop}

\medskip
Invokes tool to analyze the grammar contained in file \textit{filename}.
The \textit{filename} need not be a complete path, 
just enough to identify the file 
in current environment.
It should include file extension, if any. 


