%HHHHHHHHHHHHHHHHHHHHHHHHHHHHHHHHHHHHHHHHHHHHHHHHHHHHHHHHHHHHHHHHHHHHHHHHHH

\section{Deploying the parser\label{Deploying}}

%HHHHHHHHHHHHHHHHHHHHHHHHHHHHHHHHHHHHHHHHHHHHHHHHHHHHHHHHHHHHHHHHHHHHHHHHHH

To include the generated parser in your application, you need to include
your parser and semantics class, in the package of your choice
(specified by the \tx{-p} option on \tx{mouse.Generate}).

You also need the standard \Mouse\ package \tx{mouse.runtime}.
It is included in the \tx{JAR} file \tx{Mouse-\Version.jar},
and is also provided as a separate \tx{JAR} file \tx{Mouse-\Version.runtime.jar}.
It is sufficient to have any of them accessible via \tx{CLASSPATH}.
You may also extract the directory \tx{mouse/runtime} from any
of these \tx{JAR} files and make it accessible via \tx{CLASSPATH}.

Alternatively, you can integrate the runtime support
into your package.
You can use for this purpose the \Mouse\ tool \tx{MakeRuntime}.
For example, the command:

\small
\begin{Verbatim}[samepage=true,xleftmargin=15mm,baselinestretch=0.8]
   mouse.MakeRuntime -r my.runtime
\end{Verbatim}
\normalsize

produces in the current work directory the source code of all
runtime support classes, with declaration \tx{"package my.runtime;"}.
By specifying the option \tx{-r my.runtime} on the invocation
of \tx{Generate}, your generated parser will be an extension of \tx{my.runtime.ParserBase}
rather than \tx{mouse.runtime.ParserBase}.
The generated skeleton of your semantics class 
will be an extension of \tx{my.runtime.SemanticsBase}.\linebreak
(If you create the class on your own, you have to declare it so.)

The runtime support consists of ten classes:

\begin{tabular}{lll}
&-- \tx{ParserBase}.& \\
&-- \tx{SemanticsBase}.& \\
&-- \tx{Phrase}. \\
&-- \tx{ParserMemo} &-- extension of \tx{ParserBase}. \\
&-- \tx{ParserTest} &-- extension of \tx{ParserMemo}. \\
&-- \tx{CurrentRule} &-- interface for accessing parser stack. \\
&-- \tx{Deferred} &-- interface for deferred actions. \\
&-- \tx{Source} &-- interface of input wrappers. \\
&-- \tx{SourceFile} &-- wrapper for input from a file. \\
&-- \tx{SourceString} &-- wrapper for input from a \tx{String}.
\end{tabular}

Of these, you need \tx{ParserMemo} only if you generate the memoizing (\tx{-M})
or the instrumented (\tx{-T}) version of the parser.
You need \tx{ParserTest} only for the instrumented version.
Of \tx{SourceFile} and \tx{SourceString} you need only the one that you use.

\tx{SourceFile} assumes that the input file uses default character encoding.
You can change it by modifying \tx{SourceFile.java} in the place
indicated by a comment.

