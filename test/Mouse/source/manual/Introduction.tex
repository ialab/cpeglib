%HHHHHHHHHHHHHHHHHHHHHHHHHHHHHHHHHHHHHHHHHHHHHHHHHHHHHHHHHHHHHHHHHHHHHHHHHH

\section{Introduction}

%HHHHHHHHHHHHHHHHHHHHHHHHHHHHHHHHHHHHHHHHHHHHHHHHHHHHHHHHHHHHHHHHHHHHHHHHHH

Parsing Expression Grammar (PEG),
introduced by Ford in \cite{Ford:2004},
is a new way to specify a recursive-descent parser with limited backtracking.
\Mouse\ is a tool to transcribe PEG into an executable parser written in Java.
 

\subsubsection*{Recursive-descent parsing with backtracking}

Recursive-descent parsers have been around for a while.
Already in 1961, Lucas \cite{Lucas:1961} suggested the use of recursive procedures
that reflect the syntax of the language being parsed.
This close connection between the code and the grammar
is the great advantage of recursive-descent parsers.
It makes the parser easy to maintain and modify.

A recursive-descent parser is simple to construct
from a classical context-free grammar
if the grammar has the so-called $LL(1)$ property;
it means that the parser can always decide what to do
by looking at the next input character.
However, forcing the language into the $LL(1)$ mold 
can make the grammar -- and the parser -- unreadable.

The $LL(1)$ restriction can be circumvented by the use of backtracking.
Backtracking means that the parser proceeds by trial and error:
goes back and tries another alternative if it took a wrong decision.
However, an exhaustive search
of all alternatives may require an exponential time.
A reasonable compromise is limited backtracking, 
also called "fast-back" in \cite{Hopgood:1969}.

Limited backtracking was adopted in at least two of the early top-down designs:
the Atlas Compiler Compiler 
of Brooker and Morris \cite{Brooker:Morris:1962,Rosen:1964},  
and TMG (the TransMoGrifier) of McClure \cite{McClure:1965}.
The syntax specification used in TMG was later formalized and analyzed
by Birman and Ullman \cite{Birman:1970,Birman:Ullman:1973}.
It appears in~\cite{Aho:Ullman:1972} as "Top-Down Parsing Language" (TDPL)
and "Generalized TDPL" (GTDPL).

Parsing Expression Grammar is a development of this latter.
%
Wikipedia has an article on PEG \cite{Wiki:PEG},
and a dedicated Web site \cite{PEG} contains 
list of publications about PEGs and a link to discussion forum.

\subsubsection*{PEG programing}

Parsers defined by PEG
do not require a separate "lexer" or "scanner".
Together with lifting of the $LL(1)$ restriction,
this gives a very convenient tool when we need 
an ad-hoc parser for some application.
%
However, PEG is not well understood as a language definition tool.
The literature contains many examples of surprising behavior.
It is thus desirable to construct PEG parsers for languages
defined in a more traditional form, for example, the Extended Backus-Naur Form (EBNF).

To its external appearance, PEG is very like an EBNF grammar.
In fact, few simple typographical changes convert EBNF grammar into PEG.
It may happen that this PEG accepts exactly the language defined by EBNF.
One can then say that the EBNF grammar "is its own PEG parser".
However, in the general case the
PEG so obtained recognizes only a subset of the language defined by EBNF;
this is the effect of backtracking being limited.

A recent paper \cite{Redz:2013:FI} presents
some sufficient conditions for EBNF grammar to be its own PEG parser.
Some hints on how to construct PEG parser for an EBNF grammar 
that does not satisfy these conditions 
are given in \cite{Medeiros:2014,Redz:2014:FI}.

\subsubsection*{Packrat parsing}

Even the limited backtracking may require a lot of time.
In \cite{Ford:2002,Ford:Thesis}, PEG was introduced together with
a technique called \emph{packrat parsing}.
Packrat parsing handles backtracking
by extensive \emph{memoization}: storing all results
of parsing procedures.
It guarantees linear parsing time at a large memory cost.

(The name "packrat" comes from \emph{pack rat} -- a small rodent (\emph{Neotoma cinerea})  
known for hoarding unnecessary items.
"Memoization", introduced in \cite{Michie:1968}, is the technique  
of reusing stored results of function calls instead of recomputing them.)

Wikipedia \cite{Wiki:Rats} lists a number of generators  
producing packrat parsers from PEG.

\subsubsection*{Mouse - not a pack rat}

The amount of backtracking does not matter 
in small interactive applications
where the input is short and performance not critical.
Moreover, the usual programming languages
do not require much backtracking.
%
Experiments reported in \cite{Redz:2007:FI,Redz:2008:FI}
demonstrated a moderate backtracking activity  
in PEG parsers for programming languages Java and~C.

In view of these facts,
it is useful to construct PEG parsers
where the complexity of packrat technology is abandoned in favor
of simple and transparent design.
This is the idea of \Mouse:
a parser generator that transcribes
PEG into a set of recursive procedures that closely follow the grammar.
The name \Mouse\ was chosen in contrast to \textsl{Rats!} \cite{Grimm:2004}, 
one of the first generators producing packrat parsers.
Optionally, \Mouse\ can offer a small amount of memoization using the technique
described in \cite{Redz:2007:FI}.
Both \Mouse\ and the resulting parser are written in Java,
which makes them operating-system independent.
%
An integral feature of \Mouse\ is the mechanism for specifying
semantics (also in Java). 

\medskip
After a short presentation of PEG in the following section,
the rest of the paper has the form of a tutorial,
introducing the reader to \Mouse\ by hands-on experience.
