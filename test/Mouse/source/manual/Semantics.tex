%HHHHHHHHHHHHHHHHHHHHHHHHHHHHHHHHHHHHHHHHHHHHHHHHHHHHHHHHHHHHHHHHHHHHHHHHHH

\section{Adding semantics\label{Semantics}}

%HHHHHHHHHHHHHHHHHHHHHHHHHHHHHHHHHHHHHHHHHHHHHHHHHHHHHHHHHHHHHHHHHHHHHHHHHH

This hasn't been very exciting.
You certainly want the parser to do something more than checking the syntax.
For example, to actually compute the sum.
For this purpose, you attach "semantic actions" to rules of the grammar.

As you have seen, \Mouse\ transcribes each rule of the grammar into a parsing procedure
in the form of a Java method.
A semantic action is a Java method, written by you,
that is called by a successful parsing procedure as the last thing before it exits.
You may use this method to process the text consumed by the procedure.

In order to compute the sums, you may define a semantic action for \Number\
that will convert the digits consumed by \Number\ into an integer.
You may then define a semantic action for \Sum\ that will receive these integers,
add them, and print the result.

In most parser generators, the code for semantic actions
is inserted directly into the grammar.
\Mouse\ is a departure from that tradition.
You provide semantic actions as methods in a separate class
with any name of your choice,
for example \tx{"mySemantics"}.
It must be a subclass of \tx{mouse.runtime.SemanticsBase}
provided by \Mouse.
The superclass \tx{SemanticsBase} contains "helper methods"
that give you access to results of the parsing.
The structure of your parser together with semantic actions is as follows:

\smallskip
%----------------------------------------------------------------------
\small
\begin{center}
\begin{texdraw}
  \drawdim {mm}
  \setunitscale 0.75
  \linewd 0.2
  \textref h:L v:B
  \arrowheadtype t:F
  \arrowheadsize l:2 w:1
  %
  \move(0 0)
  \savecurrpos(*mPx *mPy)
  \rlvec(40 0) \rlvec(0 40) \rlvec(-40 0) \rlvec(0 -40)
  \move(*mPx *mPy) \rmove(2 6)  \htext{\tx{myParser}}
                   \rmove(0 -4) \htext{\tx{(generated)}}
  \move(*mPx *mPy) \rmove(20 40) \rlvec(0 10) \rmove(0 -3) \rlvec(-3 -3) \rlvec(6 0) \rlvec(-3 3)
  \move(*mPx *mPy) \rmove(6 20)   \htext{\tx{parsing}} 
                   \rmove(0 -3.5) \htext{\tx{procedure}}
  \move(*mPx *mPy) \rmove(12 26) \lpatt(2 1) \ravec(0 28) \lpatt()                 
  \move(*mPx *mPy) \rmove(29 19) \lpatt(2 1) \ravec(37 0) \lpatt()                 
  %
  \move(0 50)
  \savecurrpos(*bPx *bPy)
  \rlvec(40 0) \rlvec(0 40) \rlvec(-40 0) \rlvec(0 -40)
  \move(*bPx *bPy) \rmove(2 35)   \htext{\tx{ParserBase}}
                   \rmove(0 -4)   \htext{\tx{(Mouse)}}
  \move(*bPx *bPy) \rmove(6 15)   \htext{\tx{begin}} 
                   \rmove(0 -2.5) \htext{\tx{next}}
                   \rmove(0 -3.5) \htext{\tx{accept}}
                   \rmove(0 -2.5) \htext{\tx{etc.}}
  \move(*bPx *bPy) \rmove(20 22)  \htext{\tx{parser}}
                   \rmove(0 -2.5) \htext{\tx{stack}}
  %
  \move(60 0)
  \savecurrpos(*mSx *mSy)
  \rlvec(40 0) \rlvec(0 40) \rlvec(-40 0) \rlvec(0 -40)
  \move(*mSx *mSy) \rmove(2 6) \htext{\tx{mySemantics}}
                   \rmove(0 -4) \htext{\tx{(written by you)}}
  \move(*mSx *mSy) \rmove(20 40) \rlvec(0 10) \rmove(0 -3) \rlvec(-3 -3) \rlvec(6 0) \rlvec(-3 3)
  \move(*mSx *mSy) \rmove(8 20)   \htext{\tx{semantic}} 
                   \rmove(0 -3.5) \htext{\tx{action}}
  \move(*mSx *mSy) \rmove(14 25) \lpatt(2 1) \ravec(0 42) \lpatt() 
  \move(*mSx *mSy) \rmove(0 34) \ravec(-20 0) \rmove(18 0) \ravec(2 0)
  %
  \move(60 50)
  \savecurrpos(*bSx *bSy)
  \rlvec(40 0) \rlvec(0 40) \rlvec(-40 0) \rlvec(0 -40)
  \move(*bSx *bSy) \rmove(2 35)  \htext{\tx{SemanticsBase}}
                   \rmove(0 -4) \htext{\tx{(Mouse)}}
  \move(*bSx *bSy) \rmove(8 22) \htext{\tx{helper}} 
                   \rmove(0 -3) \htext{\tx{method}}
  \move(*bSx *bSy) \rmove(6 22) \lpatt(2 1) \ravec(-32 0) \lpatt()                 
  \move(*bSx *bSy) \rmove(0 34) \ravec(-20 0) \rmove(18 0) \ravec(2 0)
\end{texdraw}

\end{center}
\normalsize
%----------------------------------------------------------------------

All four classes are instantiated by a constructor
that is included in the generated parser.
The constructor establishes references between the
instantiated objects.

In the grammar, you indicate the presence of a semantic action by inserting its name in curly brackets
at the end of a rule:

\smallskip 
\small
\begin{Verbatim}[frame=single,framesep=2mm,samepage=true,xleftmargin=15mm,xrightmargin=15mm,baselinestretch=0.8]
   Sum    = Number ("+" Number)* !_ {sum} ;
   Number = [0-9]+ {number} ;
\end{Verbatim}
\normalsize
 
This tells that \Sum\ will have a semantic action \Suma,
and \Number\ will have a semantic action \Numbera.
You find this grammar as \tx{myGrammar.txt} in \tx{example2}.
Copy it into the \tx{work} directory (replacing the old one),
and type at the command prompt:

\small
\begin{Verbatim}[samepage=true,xleftmargin=15mm,baselinestretch=0.8]
 java mouse.Generate -G myGrammar.txt -P myParser -S mySemantics
\end{Verbatim}
\normalsize

This generates a new file \tx{myParser.java}.
By inspecting this file,
you can see that parsing procedures \Sumb\
and \Numberb\ contain calls to their semantic actions, as shown below.
The qualifier \tx{sem} is a reference to your semantics class \tx{mySemantics};
you had to supply its name in order to generate this reference.

\small
\begin{Verbatim}[frame=single,framesep=2mm,samepage=true,xleftmargin=15mm,xrightmargin=15mm,baselinestretch=0.8]
  //=====================================================================
  //  Sum = Number ("+" Number)* !_ {sum} ;
  //=====================================================================
  private boolean Sum()
    {
      begin("Sum");
      if (!Number()) return reject();
      while (Sum_0());
      if (!aheadNot()) return reject();
      sem.sum();                         // semantic action
      return accept();
    }
\end{Verbatim}
\normalsize

\small
\begin{Verbatim}[frame=single,framesep=2mm,samepage=true,xleftmargin=15mm,xrightmargin=15mm,baselinestretch=0.8]
  //=====================================================================
  //  Number = [0-9]+ {number} ;
  //=====================================================================
  private boolean Number()
    {
      begin("Number");
      if (!nextIn('0','9')) return reject();
      while (nextIn('0','9'));
      sem.number();                      // semantic action
      return accept();
\end{Verbatim}
\normalsize

Your semantics class is going to appear like this:

\small
\begin{Verbatim}[frame=single,framesep=2mm,samepage=true,xleftmargin=15mm,xrightmargin=15mm,baselinestretch=0.8]
class mySemantics extends mouse.runtime.SemanticsBase
{
  //-------------------------------------------------------------------
  //  Sum = Number ("+" Number)* !_
  //-------------------------------------------------------------------
  void sum()
    {...}
  
  //-------------------------------------------------------------------
  //  Number = [0-9]+
  //-------------------------------------------------------------------
  void number()
    {...}
\end{Verbatim}
\normalsize

If you wish, you may generate a file containing the above skeleton at the same time as you generated your parser,
by specifying the option \tx{-s} on your command:

\small
\begin{Verbatim}[samepage=true,xleftmargin=15mm,baselinestretch=0.8]
 java mouse.Generate -G myGrammar.txt -P myParser -S mySemantics -s
\end{Verbatim}
\normalsize

It replies:

\small
\begin{Verbatim}[samepage=true,xleftmargin=15mm,baselinestretch=0.8]
 2 rules
 1 unnamed
 3 terminals
 2 semantic procedures
\end{Verbatim}
\normalsize

indicating that it found two semantic actions,
and produces the skeleton file \tx{mySemantics.java} in the \tx{work} directory,
in addition to the parser file \tx{myParser.java}.

It remains now to write the two methods.
In order to do this, you need to know how to access the result of parsing. 

Behind the scenes, hidden in the service methods,
each call to a parsing procedure,
or to a service that implements a terminal,
returns a \Phrase\ object. 
The object is an instance of class \Phrase.
It describes the portion of input consumed
by the call -- the consumed "phrase".

Still behind the scenes, each parsing procedure collects 
the \Phrase\ objects returned
by the service methods and procedures that it calls.
At the end, it uses information from them
to construct its own resulting \Phrase.
The semantic action is called just after this latter
has been constructed, and can access it, together
with the \Phrase s representing the partial results.

As an illustration,
we are going to describe what happens when your parser processes 
the input \tx{"17+4711"}.

The process starts by calling \Sumb, which immediately calls \Numberb.
\Numberb, in turn, makes two calls to \tx{nextIn()} that consume,
respectively, the digits \tx{"1"} and \tx{"7"},
and return one \Phrase\ object each.
\Numberb\ (or, more precisely, the service methods called in the process) 
uses information from these objects to construct
a new \Phrase\ object that represents the text \tx{"17"}
thus consumed.
This is going to be the result of \Numberb.

All three objects are accessible to the semantic action \Numbera,
and can be visualized as follows:

%----------------------------------------------------------------------
\small
\begin{center}
\begin{texdraw}
  \drawdim {mm}
  \setunitscale 0.9
  \linewd 0.1
  \textref h:C v:C
  \arrowheadtype t:F
  \arrowheadsize l:2 w:1
  \savepos( 0  8)(*C1x *R1y)
  \savepos(30  8)(*C2x *R2y)
  \savepos(55  8)(*C3x *R3y)
  %
  \move(*C1x *R1y) \phrase{Number}{17}{}{lhs()}{}{P11}
  \move(*C2x *R1y) \phrase{[0-9]}{1}{}{rhs(0)}{}{P21}
  \move(*C3x *R1y) \phrase{[0-9]}{7}{}{rhs(1)}{}{P31}
  %
  \move(*P21lx *P21ly) \ravec(-6 0)
  %
  \move(*C2x 0)
  \savecurrpos(*cx *cy)
  \move(*cx *cy) \rlvec(8 0)
  \rmove(0 4)   \rlvec(-8 0)
  \move(*cx *cy) 
  \rmove(0 0)   \rlvec(0 4)
  \rmove(4 -4)  \rlvec(0 4)
  \rmove(4 -4)  \rlvec(0 4)
  \textref h:C v:C
  \move(*cx *cy) \rmove(2 2) 
  \rmove(0 0) \htext{\tx{1}}
  \rmove(4 0) \htext{\tx{7}}
  \rmove(4 0) \htext{\tx{+}}
  \rmove(4 0) \htext{\tx{4}}
  \rmove(4 0) \htext{\tx{7}}
  \rmove(4 0) \htext{\tx{1}}
  \rmove(4 0) \htext{\tx{1}}
\end{texdraw}
\end{center}
\normalsize
%----------------------------------------------------------------------

Each box represents one \Phrase\ object.
The text highest in the box identifies the parsing procedure or the service
that constructed it.
The text below is the consumed portion of input. 

In the semantic action,
you obtain references to the three objects by calls to helper methods
shown above the boxes.
They should be read as "left-hand side", "right-hand side element~0",
and "right-hand side element~1":
they correspond to the left- and right-hand sides of the rule \tx{"Number = [0-9]+"}.

According to the initial plan, \Numbera\ is expected
to convert the text \tx{"17"} consumed by \Numberb\
into the integer 17.
This could be done by obtaining the digits represented by \tx{rhs(0)} and \tx{rhs(1)}
and processing them to obtain the integer.
However, the integer can be obtained directly from the text represented by \tx{lhs()},
by using a method from the Java class \tx{Integer}.
To obtain the text represented by a \Phrase\ object, you apply the helper method \tx{text()} to the object.
The integer 17 is thus obtained as \tx{Integer.valueOf(lhs().text())}.

Each \Phrase\ object has a field of type \Object\ where you can insert 
the "semantic value" of the represented phrase.
It is shown here as an empty slot at the bottom of each box.
You can insert there the integer 17 as semantic value of the result.
You insert semantic value into a \Phrase\ object by applying the helper method \tx{put(...)}
to that object, so your \Numbera\ 
can be written as follows:

\small
\begin{Verbatim}[frame=single,framesep=2mm,samepage=true,xleftmargin=15mm,xrightmargin=15mm,baselinestretch=0.8]
   //-------------------------------------------------------------------
   //  Number = [0-9]+
   //-------------------------------------------------------------------
   void number()
     { lhs().put(Integer.valueOf(lhs().text())); }
\end{Verbatim}
\normalsize

After completed semantic action, \Numberb\ returns to \Sumb,
delivering this object as its result:

%----------------------------------------------------------------------
\small
\begin{center}
\begin{texdraw}
  \drawdim {mm}
  \setunitscale 0.9
  \linewd 0.1
  \move(0 0) \lowphrase{Number}{17}{Integer 17}{}{}{P11}
\end{texdraw}
\end{center}
\normalsize
%----------------------------------------------------------------------

After receiving control back, \Sumb\ proceeds, via \verb#Sum_0()#,
to call \tx{next()} and \Numberb\ that consume, respectively, \tx{"+"}
and \tx{"4711"}, and return their resulting \Phrase\ objects.
\Sumb\ uses information from them to construct
a new \Phrase\ object that represents the text \tx{"17+4711"}
thus consumed.

All four objects are accessible to the semantic action \Suma,
and can be visualized as follows:

%----------------------------------------------------------------------
\small
\begin{center}
\begin{texdraw}
  \drawdim {mm}
  \setunitscale 0.9
  \linewd 0.1
  \textref h:C v:C
  \arrowheadtype t:F
  \arrowheadsize l:2 w:1
  \savepos( 0  8)(*C1x *R1y)
  \savepos(30  8)(*C2x *R2y)
  \savepos(55  8)(*C3x *R3y)
  \savepos(80  8)(*C4x *R4y)
  %
  \move(*C1x *R1y) \phrase{Sum}{17+4711}{}{lhs()}{}{P11}
  \move(*C2x *R1y) \phrase{Number}{17}{Integer 17}{rhs(0)}{}{P21}
  \move(*C3x *R1y) \phrase{"+"}{+}{}{rhs(1)}{}{P31}
  \move(*C4x *R1y) \phrase{Number}{4711}{Integer 4711}{rhs(2)}{}{P41}
  %
  \move(*P21lx *P21ly) \ravec(-6 0)
  %
  \move(*C2x 0) \rmove(10 0)
  \savecurrpos(*cx *cy)
  \move(*cx *cy) \rlvec(28 0)
  \rmove(0 4)   \rlvec(-28 0)
  \move(*cx *cy) 
  \rmove(0 0)   \rlvec(0 4)
  \rmove(8 -4)  \rlvec(0 4)
  \rmove(4 -4)  \rlvec(0 4)
  \rmove(16 -4) \rlvec(0 4)
  \textref h:C v:C
  \move(*cx *cy) \rmove(2 2) 
  \rmove(0 0) \htext{\tx{1}}
  \rmove(4 0) \htext{\tx{7}}
  \rmove(4 0) \htext{\tx{+}}
  \rmove(4 0) \htext{\tx{4}}
  \rmove(4 0) \htext{\tx{7}}
  \rmove(4 0) \htext{\tx{1}}
  \rmove(4 0) \htext{\tx{1}}
\end{texdraw}
\end{center}
\normalsize
%----------------------------------------------------------------------

Both \Numberb\ results carry as semantic values the integers computed by \Numbera.
To obtain the semantic value of a \Phrase\ you apply 
the helper method \tx{get()} to it.
You can now easily compute the sum as follows
(you need the casts because \tx{get()} returns an \Object):
\small
\begin{Verbatim}[samepage=true,xleftmargin=15mm,baselinestretch=0.8]
     (Integer)rhs(0).get() + (Integer)rhs(2).get().
\end{Verbatim}
\normalsize
However, in the general case you may have any number $p \ge 0$
of \tx{("+"~\Number)} pairs resulting from $p$ successful calls to \verb#Sum_0()#.
Your \Suma\ must thus handle the following general situation 
with $n = 2p + 1$ \Phrase s on the right-hand side:

%----------------------------------------------------------------------
\small
\begin{center}
\begin{texdraw}
  \drawdim {mm}
  \setunitscale 0.9
  \linewd 0.1
  \textref h:C v:C
  \arrowheadtype t:F
  \arrowheadsize l:2 w:1
  \savepos( 0 0)(*C1x *R2y)
  \savepos(30 0)(*C2x *R2y)
  \savepos(55 0)(*C3x *R2y)
  \savepos(80 0)(*C4x *R2y)
  \savepos(105 0)(*C5x *R2y)
  \savepos(130 0)(*C6x *R2y)
  %
  \move(*C1x *R2y) \phrase{Sum}{}{}{lhs()}{}{P11}
  %
  \move(*C2x *R2y) \phrase{Number}{}{}{rhs(0)}{}{P21}
  \move(*C3x *R2y) \phrase{"+"}{}{}{rhs(1)}{}{P31}
  \move(*C4x *R2y) \phrase{Number}{}{}{rhs(2)}{}{P41}
  \move(*C5x *R2y) \phrase{"+"}{}{}{rhs($n$-2)}{}{P51}
  \move(*C6x *R2y) \phrase{Number}{}{}{rhs($n$-1)}{}{P61}
  %
  \move(*P21lx *P21ly) \ravec(-6 0)
  %
  \lpatt(2 1)
  \move(*C4x *R2y)\rmove(21 0) \rlvec(8 0)
  \move(*C4x *R2y)\rmove(21 15)\rlvec(8 0)
  \lpatt()
\end{texdraw}
\end{center}
\normalsize
%----------------------------------------------------------------------

To obtain the number $n$  
you use the helper method \tx{rhsSize()}.
Your method \Suma\ may thus look like this:

\small
\begin{Verbatim}[frame=single,framesep=2mm,samepage=true,xleftmargin=15mm,xrightmargin=15mm,baselinestretch=0.8]
   //-------------------------------------------------------------------
   //  Sum = Number "+" Number ... "+" Number !_
   //          0     1    2        n-2   n-1
   //-------------------------------------------------------------------
   void sum()
     {        
       int s = 0;
       for (int i=0;i<rhsSize();i+=2)
         s += (Integer)rhs(i).get();
       System.out.println(s);
     }
\end{Verbatim}
\normalsize
 
You find the complete semantics class, with the methods \Numbera\ and \Suma\
just described,
as file \tx{mySemantics.java} in \tx{example2};
copy it to the \tx{work} directory. 
To obtain the new parser, compile the semantics class   
and the parser class you have generated. 
You can try it in the same way as before.
The session may look like this:
  
\small
\begin{Verbatim}[samepage=true,xleftmargin=15mm,baselinestretch=0.8]
 javac myParser.java
 javac mySemantics.java
 java mouse.TryParser -P myParser
 > 17+4711
 4728
 > 12+2+3
 17
 >
\end{Verbatim}
\normalsize
