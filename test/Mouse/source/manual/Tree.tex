%HHHHHHHHHHHHHHHHHHHHHHHHHHHHHHHHHHHHHHHHHHHHHHHHHHHHHHHHHHHHHHHHHHHHHHHHHH

\section{Want a tree?\label{tree}}

%HHHHHHHHHHHHHHHHHHHHHHHHHHHHHHHHHHHHHHHHHHHHHHHHHHHHHHHHHHHHHHHHHHHHHHHHHH

Many existing parsers deliver their result in the form of a syntax tree,
showing how the input was parsed according to the grammar rules.
The preceding examples show that much of this information may be
uninteresting if the grammar is specified at the level
of input characters rather than "tokens" produced by a "lexer".
For example, the structure of our fractional number 
in terms of individual characters is not interesting 
as long as we only want the complete string.
For this reason, \Mouse\ does not provide automatic construction
of syntax trees.

If you wish, you may still construct syntax tree at the desired level of detail
with the help of semantic actions.
In the following, we show how the preceding example
can be modified to produce an operand-operator tree
instead of calculating the result.
Such tree can be then further processed
(for example, to generate executable code).
All you need to construct the tree is write another semantics class,
without changing the parser.

Of course, you also need to define elements of the tree to be constructed.
You find their definitions in the file \tx{Node.java} in \tx{example6}.
It defines an abstract class \tx{Node} with two subclasses:
\tx{Node.Num} and \tx{Node.Op},
to represent, respectively, a number 
and a binary operator.
%
In order to simplify the example, we do not provide separate subclasses
for different kinds of numbers and operators.
A number is stored in \tx{Node.Num} as the text string retrieved from the input;
an operator is stored in \tx{Node.Op} as one character.

The semantic values will now be partial trees, represented by their top nodes.
The semantic actions will appear like this:

\small
\begin{Verbatim}[frame=single,framesep=2mm,samepage=true,xleftmargin=15mm,xrightmargin=15mm,baselinestretch=0.8]
   //-------------------------------------------------------------------
   //  Factor = Digits? "."  Digits Space
   //              0    1(0)  2(1)   3(2)
   //-------------------------------------------------------------------
   void fraction()
     { lhs().put(new Node.Num(rhsText(0,rhsSize()-1))); }
\end{Verbatim}
\normalsize

\small
\begin{Verbatim}[frame=single,framesep=2mm,samepage=true,xleftmargin=15mm,xrightmargin=15mm,baselinestretch=0.8]
  //-------------------------------------------------------------------
  //  Sum = Sign Product AddOp Product ... AddOp Product
  //          0     1      2      3         n-2    n-1
  //-------------------------------------------------------------------
  void sum()
    {
      int n = rhsSize();
      Node N = (Node)rhs(1).get();
      if (!rhs(0).isEmpty())
        N = new Node.Op(new Node.Num("0"),'-',N);
      for (int i=3;i<n;i+=2)
        N = new Node.Op(N,rhs(i-1).charAt(0),(Node)rhs(i).get());
      lhs().put(N);
    }
\end{Verbatim}
\normalsize

(Minus as \Sign\ is represented by subtraction from 0.)

You find complete semantics in \tx{example6}.
As you can see there, the final action, \tx{result()},
saves the top node of the resulting tree in the variable \tx{tree},
local to \tx{mySemantics}.
You cannot use the standard tool \tx{mouse.TryParser} to obtain the 
tree thus stored;
you have to code invocation of the parser on your own.
It will be an occasion to learn how this is done.
The code is shown below.

\medskip
\small
\begin{Verbatim}[frame=single,framesep=2mm,samepage=true,xleftmargin=6mm,xrightmargin=6mm,baselinestretch=0.8]
   import mouse.runtime.SourceString;
   import java.io.BufferedReader;
   import java.io.InputStreamReader;

   class Try
   {
     public static void main(String argv[])
       throws Exception
       {
         BufferedReader in = new BufferedReader(new InputStreamReader(System.in));
         myParser parser = new myParser();            // Instantiate Parser+Semantics
         while (true)
         {
           System.out.print("> ");                    // Print prompt
           String line = in.readLine();               // Read line
           if (line.length()==0) return;              // Quit on empty line
           SourceString src = new SourceString(line); // Wrap up the line
           boolean ok = parser.parse(src);            // Apply Parser to it
           if (ok)                                    // If succeeded:
           {
             mySemantics sem = parser.semantics();    // Access Semantics
             System.out.println(sem.tree.show());     // Get the tree and show it
           }
         }
       }
   }
\end{Verbatim}
\normalsize

\medskip
Most of it is self-explanatory. 
The class \tx{SourceString} supplied by \Mouse\ is a wrapper for the parser's input.
It contains methods needed to access the string and to print error position in diagnostics.
After instantiating the parser, you apply it to the wrapped input
by calling the method \tx{parse()}. 
The method returns a \tx{boolean} result indicating whether the parse was successful.

When the parser is instantiated, it also instantiates its semantics class.
You can access that class by calling the method \tx{semantics()} 
on the parser.
You can then obtain the tree stored there.
This is done in the last line that exploits the method \tx{show()}
of \tx{Node} to print out the tree.

The above code is found in \tx{example6}
under the name \tx{Try.java}.
Copy the files \tx{Node.java},\linebreak
\tx{mySemantics.java}, and \tx{Try.java} 
into the \tx{work} directory and compile them.
You may now type \tx{java~Try} at the command prompt, and enter your
input after "\tx{> }".
You will get the resulting tree as a nested expression
where each operator node with its sub-nodes is enclosed by a pair of brackets \tx{[ ]}.
Your session may now look like this:

\small
\begin{Verbatim}[samepage=true,xleftmargin=15mm,baselinestretch=0.8]
 java Try
 > 1+2+3
   [[1+2]+3]
 > 2(1/10)
   [2*[1/10]]
 > (1+2)*(3-4)
   [[1+2]*[3-4]]
 > 1.23(-4+.56)
   [1.23*[[0-4]+.56]]
 >
\end{Verbatim}
\normalsize

This is not the most elegant presentation of a tree,
but the purpose is just to check what was constructed.
